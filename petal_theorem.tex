\chapter{Behaviour of analytic functions near fixed points}
\section{Behaviour near parabolic fixed points}
A point \( p \) is called a parabolic fixed point of \( f \)
if \( f(p)=p \) and \( f'(p)=e^{2\pi i t} \), where \( t \)
is a rational number.H

\begin{theorem}[\textbf{The Petal Theorem}]
Suppose that an analytic map \( f \) has the form:\[
	f(z)=z-z^{p+1}+\mathcal{O}(z^{2p+1})
\] near the origin. Then for sufficiently small \( t \),
\begin{enumerate}
	\item \( f \) maps each \( \Pi_k(t) \) into itself;
	\item \( \fof[n](z)\to 0 \) uniformly on each petal;
	\item \( \arg(\fof[n](z))\to 2k\pi /p \) locally uniformly on each petal;
	\item \( |f(z)|<|z| \) on a neighbourhood of the axis of each petal;
	\item \( f:\Pi_k(t)\to \Pi_k(t) \) is conjugate to a translation.
\end{enumerate}
\end{theorem}

\begin{proof}
	For \( 0<r_0<1 \), define the sector \( S_0 \), \[
		S_0=\{re^{i\theta}:0<r<r_0,|\theta |<\pi /p \}
	\] and the region \( W \), \[
	W=\{re^{i\theta }:r>\frac{1}{r_0^p},|\theta |>\pi\}
	.\] 
	It is clear that the map \( \sigma:z\mapsto \frac{1}{z^p} \) is a biholomorphism
	of \( S_0 \) onto \( W \) with \( \sigma^{-1}:W\to S_0 \) given by \( \sigma^{-1}(w)=1 /w^{\frac{1}{p}} \).
	The branch of \( p \)-th root that we select determines which sector of width \( 2\pi /p \), the inverse map
	maps to (the other sectors being \( S_k=\{0<r<r_0,|\theta -2k\pi /p|<\pi /p \} \).

	Now, the conjugate map of \( f \) on \( W \) is given by,\[
		g(w)=\sigma f \sigma ^{-1}(w)=f(w^{-\frac{1}{p}})^{-p}
	.\]

	This just replaces the action of \( f \) on \( S \) by \( g \) on \( W \),
	and we have the following commutative diagram\\
	???


\end{proof}
