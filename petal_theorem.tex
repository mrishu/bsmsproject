\chapter{Petal Theorem}
A point \( p \) is called a parabolic fixed point of \( f \)
if \( f(p)=p \) and \( f'(p)=e^{2\pi i t} \), where \( t \)
is a rational number.

\begin{lemma}\label{lem2.1}
	Suppose \( f \) is analytic and satisfies \[
		f(z)=z-z^{p+1}+\mathcal{O}(z^{p+2})
	\] in some neighbourhood \( N \) of the origin. Let \( \omega_1,\ldots ,\omega_p \) be the \( p \)-th
	roots of unity and let \( \eta_1,\ldots ,\eta_p \) be the \( p \)-th roots of \( -1 \).
	Then for sufficiently small \( r_0 \) and \( \theta _0 \), 
	\begin{enumerate}
		\item \( |f(z)|<|z| \) on each sector \[
				S_j=\{re^{i\theta}:0<r<r_0,|\theta -\arg(\omega_j)|<\theta _0\}
		.\] 
	\item \( |f(z)|>|z| \) on each sector \[
				\Sigma_j=\{re^{i\theta}:0<r<r_0,|\theta -\arg(\eta_j)|<\theta _0\}
	.\] 
	\end{enumerate}
\end{lemma}
\begin{proof}
	We have, \[
		f(z) /z=1-z^p+\mathcal{O}(z^{p+1})=1-z^p(1+g(z))
	,\] where \( g \) is analytic in \( N \) with \( g(0)=0 \).\\
	Now, consider the sector, \[
		S=\{z\in \mathbb{C}:|z|<\frac{1}{2};|\arg(z)|<\pi /4\}
	.\] 
	For small \( r_0 \) and \( \theta_0 \), \( z\in S_j\implies z^p(1+g(z))\in S\) and \( z\in\Sigma_j\implies -z^p(1+g(z))\in S \). This is because for small enough \( r_0 \) and \( \theta_0 \), \( z\mapsto z^p \) maps \( S_j \) onto the set \[
		S_0=\{z\in \mathbb{C}:|z|<r_0^p;|\arg(z)|<p\theta_0\}\subset S
	.\] And, \( |z^p-z^p(1+g(z))|=|z|^p |g(z)|\le M|z|^{p+1}=M(|z|^p)^{1+\frac{1}{p}} \). Hence,
	for any \( w\in S_0 \), the perturbation of any point is \( \le M|w|^{1+1/p} \).
	% TODO: Complete the proof of theorem for repelling and attracting sectors for parabolic fixed point

\end{proof}

Before stating the \emph{Petal Theorem}, which discusses the
behaviour of analytic functions near parabolic fixed points, we first define the notions
of \emph{petals}.
\begin{definition}[Petals]
	Let \( p\in \mathbb{N} \). For each \( k\in\{0,1,\ldots ,p-1\} \), define the sets as a function of a parameter \( t>0 \)
	as follows, \[
		\Pi_k(t)=\{re^{i\theta}:r^p<t(1+\cos (p\theta));|\theta-2k\pi /p|<\pi /p\}
	.\] 
	The sets \( \Pi_k(t) \) are known as Petals.
\end{definition}
We have shown a diagram of the petals \( \Pi_k(t) \) in \Cref{petals} for \( p=6 \).\\
Note that all the petals are pairwise disjoint and each petal subtends an angle of \( 2\pi /p \) at the origin.

\begin{figure}[ht]
	\centering
	\incfig{petals}
	\caption{Six petals at the origin for $p=6$.}
	\label{petals}
\end{figure}

\begin{theorem}[\textbf{The Petal Theorem}]
Suppose that an analytic map \( f \) has the form:\[
	f(z)=z-z^{p+1}+\mathcal{O}(z^{2p+1})
\] near the origin. Then for sufficiently small \( t \),
\begin{enumerate}
	\item \( f \) maps each \( \Pi_k(t) \) into itself;
	\item \( \fof[n](z)\to 0 \) uniformly on each petal;
	\item \( \arg(\fof[n](z))\to 2k\pi /p \) locally uniformly on each petal;
	\item \( f:\Pi_k(t)\to \Pi_k(t) \) is conjugate to a translation.
	\item \( |f(z)|<|z| \) on a neighbourhood of the axis of each petal;
\end{enumerate}
\end{theorem}

\begin{proof}
	For \( 0<r_0<1 \), define the sector \( S_0 \), \[
		S_0=\{re^{i\theta}:0<r<r_0,|\theta |<\pi /p \}
	\] and the region \( W \), \[
	W=\{re^{i\theta }:r>\frac{1}{r_0^p},|\theta |>\pi\}
	.\] 

\begin{figure}[ht] \centering
    \incfig{sigmamap}
    \caption{$\sigma$ is a biholomorphism from $S_0$ onto $W$.}
    \label{sigmamap}
\end{figure}

	It is clear that the map \( \sigma:z\mapsto \frac{1}{z^p} \) is a biholomorphism
	of \( S_0 \) onto \( W \) with \( \sigma^{-1}:W\to S_0 \) given by \( \sigma^{-1}(w)=1 /w^{\frac{1}{p}} \).
	Actually, \( \sigma  \) is a biholomorphism of each sector \( S_k=\{0<r<r_0,|\theta -2k\pi /p|<\pi /p \} \) onto \( W \). The branch of \( \sigma^{-1}(w)=1 /w^{\frac{1}{p}} \) that we choose determines which sector the inverse map maps to.

	Now, the conjugate map of \( f \) on \( W \) is given by,\[
		g(w)=\sigma f \sigma ^{-1}(w)=f(w^{-\frac{1}{p}})^{-p}
	.\]

	This just replaces the action of \( f \) on \( S_0 \) by \( g \) on \( W \),
	and we have the following commutative diagram:

	Now, we will use the power series expansion of \( f \) near the origin to
	get information about \( g \).\\
	First let us try to get a estimate of the power series expansion of \( f(z)^{-p} \).
	We have, \[
		f(z)=z-z^{p+1}+\mathcal{O}(z^{2p+1})=z(1-z^p+\mathcal{O}(z^{2p}))=z(1-z^p-a_0z^{2p}-a_1z^{2p+1}+-\ldots)
	.\] 
	So, \[
		\frac{1}{f(z)^p}=z^{-p}\left(\frac{1}{1-z^p-a_0z^{2p}-a_1z^{2p+1}+-\ldots}\right)^p
	.\]  
	Now, let \( \alpha(z)=z^p+a_0z^{2p}+a_1z^{2p+1}+\ldots \), 
	then for \( r_0 \) small enough
	such that \( |\alpha(z)|<1 \) on \( \{|z|<r_0\} \), we can write, \[
		\frac{1}{1-\alpha(z)}=1+\alpha(z)+\alpha(z)^2+\ldots 
	.\] 
	Therefore,
	\begin{align*}
		\frac{1}{f(z)^p}&=z^{-p}(1+\alpha(z)+\alpha(z)^2+\ldots )^p\\
						&=z^{-p}(1+pz^p+Az^{2p}+A_1z^{2p+1}+\ldots )\\
						&=\frac{1}{z^p}+p+Az^p+v(z)
	,\end{align*} where \( A \) is some constant and \( v(z) \) is holomorphic
	on \( \{|z|< r_0\} \), and for some small \( r_0>0 \),
	it satisfies \( |v(z)| \le B|z|^{p+1} \), \( B>0 \).

	Now, if \( w\in W \), then \( \sigma^{-1}(w)\in S \). Hence, by substituting \( z=\sigma^{-1}(w)=w^{-1/p} \), we have,
	\begin{align*}
		g(w)&=\sigma f\sigma^{-1}(w)\\
			&=\frac{1}{f(w^{-1/p})^{p}}\\
			&=w+p+A /w+\theta(w)
		,\end{align*} where \( |\theta (w)|=|v(w^{-1/p})|\le B|w^{-1/p}|^{p+1}=B /|w|^{1+\frac{1}{p}} \).

	\noindent Hence, we have the following estimates 
	for \( g \) which will be crucial in everything that will follow:
	\begin{align}
		g(w)&=w+p+A /w+\theta (w),\text{ where \( A \) is a constant and} \label{eqn3.1}\\
		|\theta (w)|&\le B /|w|^{1+\frac{1}{p}}, B>0 \label{eqn3.2}
	.\end{align}

	\noindent Choose any \( K \) satisfying \[
	K>\max\{ 1 /r_0^p,3(|A|+B) \}>1 \text{ (as \( r_0<1 \))}
	\]  and let, \[
		\Pi=\{x+iy:y^2>4K(K-x)\}
	.\] Clearly, \( \Pi \) is bounded by a parabola and \( \Pi\subset W \) (See \Cref{parabola}).
\begin{figure}[ht]
    \centering
    \incfig{parabola}
	\caption{$\Pi=\{(x,y):y^2>4K(K-x)\}$.}
    \label{parabola}
\end{figure}

	We have chosen this subset \( \Pi\subset W \) because we will show that \( \Pi \)
	is nothing but the conformal image of \( \Pi_0(t) \) under \( \sigma \) (for a suitable \( t \)) and
	\( g \) satisfies all the corresponding conditions that \( f \) should satisfy on \( \Pi_0(t) \) according to the
	theorem.\\
	\vspace{1pt}

	\noindent \textbf{Claim.} \( \Pi \) is the conformal image of \( \Pi_0(t) \) under \( \sigma \) for a suitable \( t \).\\
	The easiest way to see this is using polar coordinates. We write, \( z=re^{i\theta } \) for \( z\in S_0 \) 
	and \( w=\rho e^{i \phi } \) for \( w\in W \). Then, \( \rho=\frac{1}{r^p} \) and \( \phi=-p\theta  \).

	\begin{figure}[ht]
		\centering
		\incfig{parabola2}
		\caption{$\Pi=\{\rho e^{i\phi}:\rho>2K-\rho\cos\phi\}$.}
		\label{parabola2}
	\end{figure}
	
	Now, we need to express \( \Pi  \) in polar co-ordinates. To do so, we notice that points on the parabola are given by \[
		\rho \text{ (distance from focus i.e. \( 0 \))}= 2K-\rho \cos \phi \text{ (distance from directrix i.e. \( y=2K \))}
	.\] (See \Cref{parabola2}).
	Therefore, points on \( \Pi \) are given by \[
		\rho > 2K-\rho \cos \phi 
	.\] 
	Hence, \[
		\Pi=\{\rho e^{i\phi }:2K<\rho (1+\cos \phi )\}
	.\] Now, let \( \Omega=\sigma^{-1}(\Pi) \).
	Then, \( \Omega \) is given by \[
		\Omega=\{re^{i \theta }:2Kr^p<1+\cos (p\theta )\}
.\] Hence, \( \Omega=\Pi_0\left(\frac{1}{2K}\right) \).

\begin{lemma}
	\( g \) satisfies the following properties on \( \Pi \):
	\begin{enumerate}
		\item \(\Pi\) is forward invariant under \( g \).
		\item \( \gog[n](w)\to \infty \) uniformly on \( \Pi \).
		\item \( \arg(\gog[n](w))\to 0 \) locally uniformly on \( \Pi \).
		\item \( g:\Pi\to \Pi \) is conjugate to a translation.
	\end{enumerate}
\end{lemma}
\begin{proof}\text{}\\
	\noindent \textbf{\emph{1.}} We write, \[
		w=x+iy,\,\, g(w)=X+iY,\,\, A /w+\theta(w)=a+ib
	.\] 
	From \Cref{eqn3.1}, we obtain,
	\begin{align*}
		X+iY=(x+iy)+p+(a+ib)\\
	\implies X=x+p+a\text{ and }Y=y+b
	.\end{align*}
	Now, if \( w\in \Pi  \),
	\begin{align*}
		Y^2-4K(K-X)&=(y+b)^2-4K(K-x-p-a)\\
				   &=[y^2-4K(K-x)]+b^2+2yb+4K(a+p)\\
				   &> 4Kp+(2yb+4Ka)
	.\end{align*}
	Now, for \( w\in\Pi \), \( |w|>K>1 \). (It is clear for \( \re(w)>K \). For \( \re(w)\le K \), we use the polar description \( \rho>2K-\rho \cos \phi  \) to get \( |w|>2K-\re(w)\ge K \)). Hence we get,
	\begin{align}
		|w||A /w+\theta (w)|\le |w| (|A| /|w|+ B /|w|^{1+\frac{1}{p}})=|A|+B /|w|^{\frac{1}{p}}<|A|+B \label{eqn3.3}
	\end{align}
	(since for \( |w|>1 \), \( |w|^{\frac{1}{p}}>1 \)).  Therefore,
	\begin{align*}
		|2yb+4Ka|&\le 2|yb|	+4K|a|\\
				 &\le 2|y| |b|+4 K |a| \\
				 &\le 2|w| |a+ib| +4 |w| |a+ib| \\
				 &=6|w| |a+ib|\\
				 &<2K\le 2Kp
	.\end{align*}
	Therefore, we see that \( Y^2-4K(K-X)>0 \) and hence, \( g(w)\in \Pi \) for \( w\in \Pi \).
	Hence, \( \Pi \) is forward invariant under \( g \).\\
	\vspace{1pt}

	\noindent \textbf{\emph{2.}} Now, we will prove a stronger statement that for any \( t>0 \)
	\( g \) maps \( \Pi+t \) into \( \Pi+t+p /2 \). This is simply because, for \( w\in\Pi+t \),
	we have, \( y^2-4K(K+t-x)>0 \). Hence,
	\begin{align*}
		Y^2-4K(K+t+p /2-X)&=[y^2-4K(K+t-x)]+b^2+2yb+4K(a-p /2)\\
		&> 2Kp+(2yb+4Ka)\\
		&>0
	.\end{align*}
	Therefore, if \( w\in\Pi \), \( \gog[n](w)\in \Pi+np /2 \). Hence, \( |\gog[n](w)|>\sqrt{n}  \).
	This is simply because, if \( x+iy\in \Pi+np /2 \), we have \[ x^2+y^2-n>x^2+4K(K+np /2-x)-n=x^2-4Kx+(4K^2+2npK-n). \] The discriminant of this quadratic equation in \( x \) is \[ 16K^2-4(4K^2+2npK-n)=4n(1-2pK)<0 \]. Thus, \( x^2+y^2-n>0 \) for all \( x+iy\in \Pi+np /2 \).

Hence, \( \gog[n](w)\to \infty \) uniformly on \( \Pi \).\\
\vspace{1pt}

\noindent \textbf{\emph{3.}} Firstly, we see that, \[
	\gog[k](w)=g(\gog[k](w))=\gog[k](w)+p+A /\gog[k](w)+\theta (\gog[k](w))
.\] 
By expanding the first term again and again, we obtain,
\begin{align}
	\gog[n](w)&=w+np+ \sum_{k=0}^{n-1} \left(\frac{A}{\gog[k](w)}  +\theta(\gog[k](w))\right) \label{eqn3.4}
.\end{align}

Also note that form \Cref{eqn3.3}, we have, \[
	|A /w+\theta (w)|<(|A|+B) /K <\frac{1}{3}
.\] 

Let \( Q \) be a compact subset of \( \Pi \). From now, we will assume that \( w\in Q \) and we will use
\( C_1,C_2,C_2,\ldots  \) to denote positive constants which will be dependent on \( Q \).

Hence,
\begin{align*}
|g(w)|=|w+p+A /w+\theta (w)|&\ge ||w+p|-|A /w+\theta (w)| |\\
								&= |w+p|-|A /w+\theta (w)|\\
								&\ge |w|+p-\frac{1}{3}
.\end{align*}
Therefore, we obtain, \[
	|\gog[n](w)|\ge |w|+n(p-1 /3)\ge C_1+C_2n
.\] 
(Here, \( C_1=\min\{|w|:w\in Q\}>0 \) and \( C_2=p-\frac{1}{3}>0 \).)\\
Hence,
\begin{align}
	|\gog[n](w)|\ge C_2 n\label{eqn3.5}.
\end{align}
Next, with \Cref{eqn3.2}, and the above inequality, we get,
\begin{align}
	|\theta (\gog[n](w)|\le B /|\gog[n](w)|^{1+\frac{1}{p}}\le C_3 /n^{1+\frac{1}{p}}\label{eqn3.6}
.\end{align}

Finally, combining the above two inequalities and \Cref{eqn3.4}, we obtain,
\begin{align*}
	|\gog[n](w)-np|&\le |w|+|A /w+\theta (w)|+\frac{|A|}{C_2}\sum_{k=1}^{n-1} \frac{1}{k}+C_3\sum_{k=1}^{n-1} \frac{1}{n^{1+\frac{1}{p}}}\\
				   &< C_4+C_5\sum_{k=1}^{n} \frac{1}{k}
.\end{align*}
(Here, \( C_4=\max\{|w|:w\in Q\}+\frac{1}{3}+C_3\sum_{n=1}^{\infty}  1 /n^{1+\frac{1}{p}} \) and \( C_5=|A| /C_2\).)\\
We can select \( C_6 \) large enough such that
\begin{align}
	|\gog[n](w)-np|<C_6 \log n \label{eqn3.7}
.\end{align}

\begin{remark}
	The above inequality follows from the fact that, if \( H_n=\sum_{k=1}^n \frac{1}{k} \), then \( H_n-\log n\to \gamma \).
	(\( \gamma \) is known as the Euler's constant).
	So, we have that
	\begin{align*}
		P+QH_n &=P+Q(\log n+\gamma +\epsilon_n), \text{where \( \epsilon_n\to 0 \)}\\
			   &\le Q\log n+(P+Q\max\{\epsilon_n\}+Q\gamma)\\
			   &=Q\log n+R\\
			   &<S \log n
	\end{align*} for \( S \) large enough.
\end{remark}

\begin{figure}[ht]
    \centering
    \incfig{argogn}
	\caption{$|\arg(\gog[n](w))|\le \sin^{-1}(\frac{C_6 \log n}{np})$.}
    \label{argogn}
\end{figure}
From, \( |\gog[n](w)-np|<C_6\log n \), it follows that \( |\arg(\gog[n](w)|<\sin ^{-1}\left(\frac{C_6\log n}{np}\right) \)
for \( n \) large enough. Hence, \( \arg(\gog[n](w))\to 0 \) uniformly on \( Q \), and consequently, locally uniformly on \( \Pi \).\\
\vspace{1pt}

\noindent\textbf{\emph{4.}} Define, \[
	u_n(w)=\gog[n](w)-np-(A /p)\log n
.\] 
\noindent\textbf{Claim.} \( u_n(w) \) converges locally uniformly on \( \Pi \) to a holomorphic function \( u \), that is one-to-one
on \( \Pi \).
\begin{align*}
	u_{n+1}(w)-u_n(w)&=[\gog[n+1](w)-\gog[n](w)]-p-(A /p)\log \left(\frac{n+1}{n}\right)
.\end{align*}
From \Cref{eqn3.2}, we obtain, 
\begin{align*}
	u_{n+1}(w)-u_n(w) &= \!\begin{multlined}[t]
							[\gog[n](w)+p+A /\gog[n](w)+\theta(\gog[n](w))-\gog[n](w)]\\
							 -p-(A /p)\log (1+1 /n)
						 \end{multlined}\\
					  &= A /\gog[n](w)+\theta (\gog[n](w))- (A /p)\log (1+1 /n)\\
					  &= A(1 /\gog[n](w)-1 /np)+\theta (\gog[n](w))+(A /p)(1 /n-\log (1+1 /n))
.\end{align*}
Now, let \( Q \) is a compact subset of \( \Pi \) and \( w\in Q \).
We need to prove that \( u_n \) converges uniformly in \( Q \). From the above equation, to prove
that \( u_n \) converges uniformly in \( Q \), we need to show that each of the following series converges uniformly in \( Q \):\[
	\sum_n |1 /\gog[n](w)-1 /np|,\,\,\sum_n|\theta (\gog[n](w)),\,\,\sum_n|1 /n-\log (1+1 /n)|
.\] 
Let us look at the first series. We have, (using \Cref{eqn3.5,eqn3.7}) \[
	|1 /\gog[n](w)-1 /np|=\frac{|\gog[n](w)-np|}{|\gog[n](w)|np}\le \frac{C_6 \log n}{C_2 n^2 p}=C_7\log n /n^2
.\] 
(Here \( C_7=C_6 /(pC_2) \)).

\noindent From \Cref{eqn3.6}, it is clear that \( \sum_n |\theta(\gog[n](w))| \) converges.

\noindent Now, \( 0<x-\log (1+x)\le   x^2 \) for \( x>0 \). \\
This is because, it is zero at \( x=0 \)
and \( \frac{d}{dx}(x-\log (1+x))=1-\frac{1}{1+x}>0 \) for \( x>0 \).\\
Also, \( x^2-x+\log (1+x) \) is zero at \( x=0 \) 
and \( \frac{d}{dx}(x^2-x+\log (1+x))=2x-1+\frac{1}{1+x}>0 \) for \( x>0 \).

Putting \( x=\frac{1}{n} \), we get, \[
	|1 /n-\log (1+ 1 /n)|<1 /n^2
.\] 
Therefore, \( u_n\) converges locally uniformly to some holomorphic function \( u \) on \( \Pi \).\\
Now, from \( u_n(w)=\gog[n](w)-np-(A /p)\log n \), we get that, 
\begin{align*}
	(n+1)p+(A /p)\log (n+1)+u_{n+1}(w)&=\gog[n+1](w)\\
									  &=\gog[n](g(w))\\
									  &=np+(A /p)\log n+u_n(g(w))\\
	\implies p+(A /p)\log (1+1 /n)+u_{n+1}(w)&=u_n(g(w))
.\end{align*}
Taking limit \( n\to \infty \), we get, \[
	p+u(w)=u(g(w))
.\] 
Since \( f \) is injective near the origin, \( g \) is injective on \( \Pi \),
(if \( K \) is chosen large enough). Therefore, \( \gog[n] \) is injective on \( \Pi \)
and hence, so is \( u_n \). By Hurwitz Theorem, \( u \) is either injective or constant, but it
is clearly not a constant since it satisfies the above equation.

This shows that \( g:\Pi\to \Pi \) is conjugate to the map \( z\mapsto z+p \) of \( u(\Pi) \) into
itself.
\end{proof}

Coming back to our original theorem, we see that our original theorem is also 
proved as we had just replaced the action of \( f \) on \( \Pi_0 \) by the action of 
its conjugate \( g \) on \( \Pi \) and we just proved all the parts of the theorem that
the conjugate of \( f \), i.e. \( g \) must satisfy.

From, \( g=\sigma f\sigma^{-1} \), we get,
\( \gog[n]=\sigma \fof[n]\sigma^{-1}\implies\gog[n]\sigma= \sigma\fof[n]\). 
Writing, \( \sigma(z)=w \), we have,
\begin{align}
	\gog[n](w)=\frac{1}{\fof[n](z)^p}\implies\gog[n](w)(\fof[n](z))^p=1\label{eqn8}
.\end{align}
\begin{enumerate}
	\item Since, \( g \) maps \( \Pi \) into itself, \( f \) maps \( \Pi_0 \) into itself.
	\item Now, since \( |\gog[n](w)|>\sqrt{n}  \), \( |\fof[n](z)|<\frac{1}{n^{1/2p}} \) from \Cref{eqn8}. Hence, \( \fof[n](z)\to 0 \) uniformly on \( \Pi_0 \).
	\item Also, \( \arg(\fof[n](z))=\left(-\frac{1}{p}\right)\arg(\gog[n](w)) \) from \Cref{eqn8}. Since, \( \arg(\gog[n](w))\to 0 \) locally uniformly on \( \Pi \), \( \fof[n](z)=\left(-\frac{1}{p}\right)\arg(\gog[n](w))\to 0 \) locally uniformly on \( \Pi_0 \).
	\item Since, \( g:\Pi\to \Pi \) is conjugate to a translation, and \( g \) is conjugate to \( f \), \( f:\Pi_0\to \Pi_0 \) is also conjugate to a translation.
	\item It is immediate from \Cref{lem2.1} that \( |f(z)|<|z| \) on the axis of \( \Pi_0 \).
\end{enumerate}

\end{proof}

\begin{theorem}
	Suppose that \( f \) has the power series expansion near \( 0 \) as,\[
		f(z)=z+az^{p+1}+\mathcal{O}(z^{p+2})
	.\] Then, \( f \) is conjugate near \( 0 \) to a function \[
	F(z)=z-z^{p+1}+\mathcal{O}(z^{2p+1})
	.\] 
\end{theorem}
\begin{proof}
	First, we conjugate \( f \) by the map \( z\mapsto \lambda z \), where \( \lambda^p=a \). Then, we get that \( f \) is conjugate to the map, \[ \tilde{f}=\lambda f(z /\lambda)=\lambda z /\lambda+\lambda a z^{p+1} /\lambda^{p+1}+\mathcal{O}(z^{p+2})=z+z^{p+1}+\mathcal{O}(z^{p+2}). \]
	We will now proceed via induction over a finite number of steps. Let, \[
		f_k(z)=z+z^{p+1}+bz^{p+k+1}+\ldots, b\neq 0 
	.\] Here \( k\ge 1 \). Also if \( k\ge p \), our theorem is proved. Hence, we assume, \( 1\le k<p \).\\
	Now, define the polynomial, \[
	\sigma(z)=z+\alpha z^{k+1}
	,\] where \( \alpha=\frac{b}{p-k} \) and let \( \sigma^{-1} \) be its inverse near \( 0 \) (We can do this because \( \sigma'(0)=1 \)). 

	Now, we will show that we obtain \( f_r \) (for some \( r\ge k+1 \)) by conjugating 
	\( f_k \) with \( \sigma \).
	Hence, let \[
		g=\sigma f_k\sigma^{-1}
		\] and we need to show that \( g=f_r \) (for some \( r\ge k+1 \)). Since, \( g'(0)=f_k'(0)=1 \), we let, \[
	g(z)=z+\sum_{m=2}^\infty a_mz^m
	.\] 
	Now, we will use the identity, \( g\sigma=\sigma f_k \).
	\begin{align*}
		\sigma f_k(z)&=(z+z^{p+1}+bz^{p+k+1}+\ldots )+\alpha(z+z^{p+1}+bz^{p+k+1})^{k+1}\\
			   &=z+\alpha z^{k+1}+ z^{p+1}+(b+\alpha(k+1))z^{p+k+1}+\mathcal{O}(z^{p+k+2})\\
			   &=z+\alpha z^{k+1}+ z^{p+1}+\alpha(p+1)z^{p+k+1}+\mathcal{O}(z^{p+k+2})
	.\end{align*}
	The last equality follows because, \[ \alpha(p-k)=b\implies \alpha(p+1)-\alpha(k+1)=b\implies\alpha(p+1)=b+\alpha(k+1). \]
	Now,
	\begin{align*}
		g\sigma(z)&=(z+\alpha z^{k+1})+\sum_{m=2}^\infty a_m(z+\alpha z^{k+1})^m\\
				  &=z+\alpha z^{k+1}+\sum_{m=2}^{p+k+1}a_m(z+\alpha z^{k+1})^m+\mathcal{O}(z^{p+k+2})
	.\end{align*}
	Now, equating \( \sigma f_k(z)=g\sigma(z) \), we get,
	\begin{align*}
		z^{p+1}+\alpha(p+1)z^{p+k+1}+\mathcal{O}(z^{p+k+2})=\sum_{m=2}^{p+k+1} a_m(z+\alpha z^{k+1})^m+\mathcal{O}(z^{p+k+2})
	.\end{align*}
	Firstly, we see that on the right hand side, the coefficient of \( z^2 \) will be \( a_2 \),
	the coefficient of \( z^3 \) will be some linear combination of \( a_2 \) and \( a_3 \), the
	coefficient of \( z^4 \) will be some linear combination of \( a_2,a_3 \) and \( a_4 \) and so on upto the coefficient of \( z^p \) will be some linear combination of \( a_2,a_3,\ldots a_p \).
	Since, the coefficient of \( z^2,\ldots ,z^p \) is zero on the left hand side, it follows that \( a_2=a_3=\ldots, a_p=0 \). (This argument follows assuming \( p\ge 2 \), but if \( p=1 \) the coefficient of \( z^p=z \) i.e. \( a_1 \) is automatically \( 0 \)).

	Hence, now we have,
	\begin{align*}
		z^{p+1}+\alpha(p+1)z^{p+k+1}+\mathcal{O}(z^{p+k+2})&=\sum_{m=p+1}^{p+k+1} a_m(z+\alpha z^{k+1})^m+\mathcal{O}(z^{p+k+2})\\
														   &=a_{p+1}z^{p+1}+\ldots +a_{p+k+1}z^{p+k+1}+\\&\,\,\,a_{p+1}\alpha(p+1)z^{p+k+1}+\mathcal{O}(z^{p+k+2})
	.\end{align*}
	Therefore, we obtain \[
		a_{p+1}=1,a_{p+2}=\ldots =a_{p+k}=0\text{ and }a_{p+k+1}+a_{p+1}\alpha(p+1)=\alpha(p+1)
	.\] 
	Hence, \( a_{p+k+1}=0 \). This gives that \( f_k \) is conjugate to the map \[
		g(z)=z+z^{p+1}+\mathcal{O}(z^{p+k+2})
	.\] 
	Thus, \( g=f_r \) for some \( r\ge k+1 \). Continuing the induction process, we get that \( f \)
	is conjugate near \( 0 \) to a map \[
		z\mapsto z+z^{p+1}+\mathcal{O}(z^{2p+1})
	.\] 
	Now, we an again conjugate this map with the map, \( z\mapsto \lambda z \), where \( \lambda^p=-1 \) to get that \( f \) is conjugate to a map, \[
		F(z)=z-z^{p+1}+\mathcal{O}(z^{2p+1})
	.\] 
\end{proof}
