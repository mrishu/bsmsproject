\chapter{Introduction}
\section{Equicontinuity and Normality}

\section{Completely Invariant Components}
A domain \( D \) is called:
\begin{itemize}
	\item \emph{forward invariant} under the map \( f \) if \( f(D)=D \).
	\item \emph{backward invariant} under the map \( f \) if \( f^{-1}(D)=D \).
	\item \emph{completely invariant} under the map \( f \) if it is both forward
		and backward invariant under \( f \) i.e. \( f(D)=D \) and \( f^{-1}(D)=D \).
\end{itemize}

\begin{definition}[Connectivity]
	The connectivity of a domain \( D \subset \Cinf \) is defined as the number
	of components of \( \partial D \).
\end{definition}

\begin{theorem}\label{thm1.1}
	The following are equivalent for a domain \( D\subset \Cinf \):
	\begin{enumerate}
		\item \( D \) is simply connected.
		\item \( D^c \) is connected.
		\item \( \partial D \) is connected or \( c(D)=1 \).
	\end{enumerate}
\end{theorem}

\begin{theorem}\label{thm1.2}
	If \( R \) is a rational function, with \( \deg(R)\ge 2 \), and \( F_0 \) is
	a completely invariant Fatou component of \( R \), then:
	\begin{enumerate}
		\item \( \partial F_0=J \).
		\item \( F_0 \) is simply connected or infinitely connected.
		\item All other components of \( F \) are simply connected.
		\item \( F_0 \) is simply connected \( \iff \) \( J \) is connected.
	\end{enumerate}
\end{theorem}

\begin{theorem}\label{thm1.3}
\( \partial R(U) \subset R(\partial U) \)
\end{theorem}

\begin{lemma}\label{lem1.1}
	For a rational map \( R \), if \( F_1 \) and \( F_2 \) are two Fatou components and \( R \)
	maps a point of \( F_1 \) to a point of \( F_2 \), then \( R(F_1)=F_2 \).
\end{lemma}

\begin{theorem}\label{thm1.4}
	The unbounded Fatou component of a polynomial \( P \), i.e. the Fatou component containing \( \infty \)
	is a completely invariant Fatou component. It is denoted by \( F_\infty(P) \) or simply \( F_\infty \) when
	the context is clear.
\end{theorem}

\begin{theorem}[\textbf{Vitali's Theorem}]

\end{theorem}

\begin{lemma}\label{lem1.2}
	If \( \alpha \) is a (super)-attracting fixed point of a rational map \( R \) and \( F_\alpha \) is the
	Fatou component containing \( \alpha \) then \( \ror[n](z)\to \alpha \)
	locally uniformly in \( F_\alpha \).
\end{lemma}

\begin{theorem}[\textbf{Riemann-Hurwitz Formula}]
	Let \( F_0 \) and \( F_1 \) be components of the Fatou set \( F \) of a rational map \( R \)
	and \( R \) maps \( F_0 \) into \( F_1 \). Then, for some integer \( m \), \( R \) is an \( m \)-fold
	map of \( F_0 \) onto \( F_1 \) and \[
		\chi(F_0)+\delta_R(F_0)=m\chi(F_1)
	.\] 
\end{theorem}

\section{Some properties of the Julia Sets}
Let \( J \) denote the Julia set for a rational map \( R \) with \( \deg(R)\ge 2 \). Then we have the following properties:
\begin{theorem} 
	\( J \) is infinite.
\end{theorem}

\begin{theorem}[\textbf{Minimality of \( J \)}]
	
\end{theorem}

\begin{theorem}
	\( J \) is a perfect set, and hence, uncountable.
\end{theorem}
