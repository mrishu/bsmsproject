\chapter{Introduction}
\section{Spherical and Chordal metric}

\section{Definition of Fatou and Julia sets in terms of equicontinuity}
\begin{definition}[\textbf{Fatou and Julia Sets}]
	Let \( R \) be a non-constant rational function. The Fatou set of \( R \) denoted by \( F(R) \) is 
	the maximal open subset of \( \Cinf \) on which \( \{\ror[n]\}\) is equicontinuous. The Julia set of \( R \),
	denoted by \( J(R) \) is the complement of \( F(R) \) in \( \Cinf \).
\end{definition}
By definition, \( F(R) \) is open and \( J(R) \) is compact.\\
They are denoted by simply \( F \) or \( J \) when the context is clear.

\section{Completely Invariant Components}
If \( f:X\to X \), then a subset \( D\subset X \) is:
\begin{itemize}
	\item \emph{forward invariant} under the map \( f \) if \( f(D)=D \).
	\item \emph{backward invariant} under the map \( f \) if \( f^{-1}(D)=D \).
	\item \emph{completely invariant} under the map \( f \) if it is both forward
		and backward invariant under \( f \) i.e. \( f(D)=D \) and \( f^{-1}(D)=D \).
\end{itemize}
Note that if \( f \) is surjective, i.e. \( f(X)=X \), then backward invariance implies complete invariance.
This is because, \( f(f^{-1}(D))=D \) if \( f \) is surjective. Hence, if \(f^{-1}(D)=D  \), we have \( f(D)=D \) i.e.
forward invariance.

\begin{theorem}
	If \( f:X\to X \) be a continuous, open and surjective map of a topological space \( X \) onto itself.
	If \( D\subset X \) is completely invariant under \( f \), then so are the complement \( X\bs D \), the interior \( D^0 \), 
	the boundary \( \partial D \) and the closure \( \overline{D}  \).
\end{theorem}
\begin{proof}
	Firstly, note that it is enough to prove backward invariance since \( f \) is surjective.\\
	It is trivial to see that \( X\bs D \) is completely invariant. 

	Now, since \( f \) is a continuous map,
	\( f^{-1}(D^0) \) is an open subset of \( f^{-1}(D)=D \). Hence, \( f^{-1}(D^0) \subset D^0 \). Now, since \( f \) is an open map,
	\( f(D^0) \) is an open subset of \( f(D)=D \). Hence, \( f(D^0) \subset D^0 \implies D^0 \subset f^{-1}(f(D^0)) \subset f^{-1}(D^0) \).
	Hence, \( f^{-1}(D^0)=D^0 \) and hence, \( D^0 \) is completely invariant.
	
	From the general fact for continuous maps, \( \overline{ f^{-1}(A)}\subset f^{-1}(\overline{A})  \). Hence, \( \overline{D}\subset f^{-1}(\overline{D})   \). Now, let \( x\in f^{-1}(\overline{D})  \) (or \( f(x)\in \overline{D}  \)). If \( x\not\in \overline{D}  \), then there exists and open set around \( x \), say \( U \) such that \( U\cap D=\phi \). Since \( f \) is an open map, \( f(U) \) is an open set containing \( f(x) \). Since, \( f(x)\in \overline{D}  \), \( f(U)\cap D\neq \phi \). But since, \( f^{-1}(D)=D \), \( f^{-1}(f(U)\cap D)\subset D \). But, \( f^{-1}(f(U)\cap D)\cap U\neq \phi \implies D\cap U\neq \phi\), which is a contradiction. Hence, \( \overline{D}=f^{-1}(\overline{D})   \). Hence, \( \overline{D}  \) is also completely invariant.

	Consequently, \( \partial D=\overline{D}\bs D^0  \) is also completely invariant.
\end{proof}

\begin{theorem}
	For any rational function \( R \), the Fatou and Julia sets of \( R \) i.e. \( F(R) \) and
	\( J(R) \) are completely invariant.
\end{theorem}
\begin{proof}
	First note that it is enough to prove only backward invariance because \( R \) is surjective.
	Also, we will only prove the complete invariance of \( F(R) \), the complete invariance of \( J(R) \) then follows
	from above theorem. We will use \( F \) to denote \( F(R) \).

	Let \( z_0\in R^{-1}(F) \) and let \( w_0=R(z_0)\in F \). By equicontinuity, for any \( \epsilon>0 \), \( \exists \delta>0 \) such
	that if \( \sigma(z,z_0)<\delta \), then for all \( n \in \mathbb{N} \), \( \sigma(\ror[n](w),\ror[n](w_0))<\epsilon \). By continuity of \( R \),
	there exists \( \delta'>w_0 \) such that if \( \sigma(z,w_0)<\delta' \), then \( \sigma(R(z),w_0)<\delta \) and hence,
	\( \sigma(\ror[n+1](z),\ror[n+1](z_0))<\epsilon \) for all \( n\in \mathbb{N} \). Hence, \( \{\ror[n+1]:n \in \mathbb{N}\} \) is equicontinuous 
	at \( z_0 \) and hence, so is \( \{\ror[n]:n \in \mathbb{N}\} \). Therefore, \( z_0 \in F \) and \( R^{-1}(F) \subset F \). 

	Now, let \( z_0 \in F \). To prove that \( z_0\in R^{-1}(F) \), we need to prove that \( R(z_0)\in F \). Let \( w_0=R(z_0) \).
	We have by equicontinuity, that for any \(\epsilon>0  \), \( \exists \delta>0 \) such that for all \( n \in  \mathbb{N} \), if \( \sigma(z,z_0)<\delta \), then \( \sigma(\ror[n+1](z),\ror[n+1](z_0))<\epsilon \). Now, \( N=\{z:\sigma(z,z_0)<\delta\} \) is an open set containing \( z_0 \)
	and hence, \( R(N) \) is an open set containing \( w_0 \). Now, if \( w\in R(N) \) then \( w=R(z) \) for some \( z\in N \). Hence, \[
		\sigma(\ror[n](w),\ror[n](w_0))=\sigma(\ror[n+1](z),\ror[n+1](z_0))<\epsilon
	.\] Hence, \( z_0\in R^{-1}(F) \) and \( F \subset R^{-1}(F) \).

Therefore, \( R^{-1}(F)=F \) and \( F(R) \) is completely invariant.
\end{proof} 

\begin{lemma}\label{thm1.3}
For any rational map \( R \) and a domain \( U \subset  \Cinf\), \( \partial R(U) \subset R(\partial U) \).
\end{lemma}
\begin{proof}
	Let \( w_0\in \partial R(U) \) such that it is approximated by \( R(z_n) \) for \( (z_n)_{n=1}^\infty \subset U \).
	Now, assume \( z_n\to z_0 \) (after taking a subsequence). Now, \( z_0 \) cannot lie in \( U \), otherwise \( R(z_0)=w_0 \in R(U) \).
	Since, \( R \) is an open map, \( R(U) \) is an open set and is disjoint from \( \partial R(U) \). Hence, \( z_0\in \partial U \)
	and \( R(z_0)=w_0 \in R(\partial U) \). Therefore, \( \partial R(U) \subset R(\partial U) \).
\end{proof}

\begin{lemma}\label{lem1.1}
	For a rational map \( R \), if \( F_1 \) and \( F_2 \) are two Fatou components and \( R \)
	maps a point of \( F_1 \) to a point of \( F_2 \), then \( R(F_1)=F_2 \).
\end{lemma}
\begin{proof}
Clearly, \( R(F_1)\subset F_2 \) because of forward invariance of \( F \) under \( R \) and since \( F_1 \) and \( F_2 \)
are connected components of \( F \). If \( R(F_1)\neq F_2 \), then \( \exists z \in \partial F_1 \) such that \( R(z)\in F_2 \) and
this is not possible as \( z\in \partial F_1\implies z\in J \) and \( J \) is completely invariant. Hence, \( R(F_1)=F_2 \).
\end{proof}

\begin{theorem}\label{thm1.4}
	The unbounded Fatou component of a polynomial \( P \), i.e. the Fatou component containing \( \infty \)
	is a completely invariant Fatou component. It is denoted by \( F_\infty(P) \) or simply \( F_\infty \) when
	the context is clear.
\end{theorem}
\begin{proof}
	First note that since \( P(\infty)=\infty \), we have \( P(F_\infty)=F_\infty \) by the above lemma. Hence, \( F_\infty \subset P^{-1}(F_\infty) \).
	Now assume, some point \( z_0\in P^{-1}(F_\infty) \) but \( z_0\not\in F_\infty \). By backward invariance of \( F \),  \( z_0\in F' \),
	where \( F' \) is some other Fatou component. Again, \( P(F')=F_\infty \) by above lemma. But for polynomials, we have \( P^{-1}(\infty)=\{\infty\}\). Hence, \( \infty \in F' \) and \( F' \) must be \( F_\infty \) itself. Therefore, \( P^{-1}(F_\infty)=F_\infty \) and \( F_\infty \) is completely invariant under \( P \).
\end{proof}

\section{Equicontinuity and Normality}

\begin{definition}[Connectivity]
	The connectivity of a domain \( D \subset \Cinf \) is defined as the number
	of components of \( \partial D \).
\end{definition}

\begin{theorem}\label{thm1.1}
	The following are equivalent for a domain \( D\subset \Cinf \):
	\begin{enumerate}
		\item \( D \) is simply connected.
		\item \( D^c \) is connected.
		\item \( \partial D \) is connected or \( c(D)=1 \).
	\end{enumerate}
\end{theorem}

\begin{theorem}\label{thm1.2}
	If \( R \) is a rational function, with \( \deg(R)\ge 2 \), and \( F_0 \) is
	a completely invariant Fatou component of \( R \), then:
	\begin{enumerate}
		\item \( \partial F_0=J \).
		\item \( F_0 \) is simply connected or infinitely connected.
		\item All other components of \( F \) are simply connected.
		\item \( F_0 \) is simply connected \( \iff \) \( J \) is connected.
	\end{enumerate}
\end{theorem}


\begin{theorem}[\textbf{Vitali's Theorem}]

\end{theorem}

\begin{lemma}\label{lem1.2}
	If \( \alpha \) is a (super)-attracting fixed point of a rational map \( R \) and \( F_\alpha \) is the
	Fatou component containing \( \alpha \) then \( \ror[n](z)\to \alpha \)
	locally uniformly in \( F_\alpha \).
\end{lemma}

\begin{theorem}[\textbf{Riemann-Hurwitz Formula}]
	Let \( F_0 \) and \( F_1 \) be components of the Fatou set \( F \) of a rational map \( R \)
	and \( R \) maps \( F_0 \) into \( F_1 \). Then, for some integer \( m \), \( R \) is an \( m \)-fold
	map of \( F_0 \) onto \( F_1 \) and \[
		\chi(F_0)+\delta_R(F_0)=m\chi(F_1)
	.\] 
\end{theorem}

\section{Some properties of the Julia Sets}
Let \( J \) denote the Julia set for a rational map \( R \) with \( \deg(R)\ge 2 \). Then we have the following properties:
\begin{theorem} 
	\( J \) is infinite.
\end{theorem}

\begin{theorem}[\textbf{Minimality of \( J \)}]
	
\end{theorem}

\begin{theorem}
	\( J \) is a perfect set, and hence, uncountable.
\end{theorem}
